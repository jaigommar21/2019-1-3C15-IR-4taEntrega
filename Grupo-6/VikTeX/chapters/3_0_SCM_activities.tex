
\chapter{Actividades del SCM}
\section{Nomenclatura de Elementos}
\label{intro}
En esta sección se especifican la identificación y descripción única de cada elemento de configuración.
Además se especifica cómo se distinguirán las diferentes versiones de cada elemento.
Para todos los elementos de configuración se les deberá agregar, después del nombre del mismo, información acerca del grupo al que corresponde el elemento y la versión del mismo.
El formato para esta nomenclatura es:


Nomenclatura-vX.Y.Z

\begin{itemize}
\item Nomenclatura es la especificada más abajo para cada elemento.
\item X es un número de 1 dígito que identifica al grupo.
\item Y indica la versión del elemento de configuración o entregable.
\item Extensión indica la extensión del elemento de configuración o entregable.
\end{itemize}


Para los entregables, se deberá identificar a que Fase e iteración corresponden en forma manual. Esto es: para los elementos bajo control de configuración se los almacenará de forma que se puedan recuperar dada la Fase e iteración a la que corresponden, y para los elementos que no se encuentran bajo control de configuración podrán ser almacenados por ejemplo en carpetas que identifiquen la Fase e iteración a la que pertenecen.
Se indica la siguiente nomenclatura para cada entregable en el modelo de proceso, según la disciplina (en caso que exista algún elemento de configuración que se agregue a los que se detallan abajo, se deberá incluir en las tablas siguientes de acuerdo a la disciplina a la que pertenece, indicando la nomenclatura usada):


\section{Control de Configuración}
\label{intro}
Para realizar el cambio al sistema correspondiente deseado, se debe proceder a presentar la solicitud de cambio donde se especifica la fecha, la persona que solicita el cambio, el numero de solicitud 
correspondiente, la descripción de la solicitud, los impactos que tendrá sobre el proyecto cómo 
coste, tiempo, alcance o calidad, acciones preventivas para minimizar el impacto, estas serán revisadas por el comité de control de cambios, quienes son los autorizados de aprobar o rechazar dicha solicitud.



\section{Aprobación o Desaprobación de Cambios}
\label{intro}
Para la aprobación o rechazo de cambios en el sistema y la aseguración de la implementación de los cambios
aprobados, se formará el Comité de Control de Configuración (CCC).
La composición del CCC estará conformado por:
\begin{itemize}
\item Jefe de proyecto
\item Analista 
\item Programador
\end{itemize}



\cleardoublepage